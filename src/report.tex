\documentclass[10pt,oneside]{report}

% Encoding and font packages
\usepackage[utf8]{inputenc}
\usepackage[T1]{fontenc}
\usepackage{lmodern}
\usepackage[nottoc]{tocbibind} % If you don't want the main TOC itself listed

% Spacing and layout (optional, adjust as needed)
\usepackage{setspace}
\onehalfspacing

% Hyperlinks in the document
\usepackage{hyperref}
\hypersetup{
    colorlinks=true,
    linkcolor=blue,
    filecolor=magenta,      
    urlcolor=cyan,
    citecolor=blue
}

% Citation management (using BibTeX)
\usepackage{natbib}

\usepackage[colorinlistoftodos,prependcaption,textsize=tiny]{todonotes}

% Begin document
\begin{document}

% -------------------------
% Title Page
% -------------------------
\begin{titlepage}
    \centering
    % Title (two lines)
    {\LARGE TODO TITLE\\\par}
    \vspace{2cm}
    % Subtitle/description
    {\large A report submitted to the University of Manchester for the degree of
    Bachelor of Science in the Faculty of Science and Engineering\par}
    \vspace{2cm}
    \vspace{1cm}

    {\large Author: Sai Putravu\par}
    {\large Student id: 10829976\par}
    {\large Supervisor: TODO\par}
    \vfill
    % Department
    % Date
    {\large 2025\par}
    \vspace{1cm}
    {\large School of Computer Science\par}
\end{titlepage}

% Roman page numbering for preliminary pages
\pagenumbering{roman}

% -------------------------
% Table of Contents
% -------------------------
\tableofcontents
\clearpage

% -------------------------
% List of Figures
% -------------------------
\listoffigures
\clearpage

% -------------------------
% List of Tables
% -------------------------
\listoftables
\clearpage


% -------------------------
% Abbreviations and Acronyms
% -------------------------
% \phantomsection
\addcontentsline{toc}{chapter}{Abbreviations and Acronyms}
\chapter*{Abbreviations and Acronyms}

\begin{table}[ht]
    \centering
    \begin{tabular}{ll}
        % \textbf{AI}  & Artificial Intelligence \\
        % \textbf{ML}  & Machine Learning \\
        % \textbf{CPU} & Central Processing Unit \\
        % \textbf{GPU} & Graphics Processing Unit \\
        % \textbf{BSc} & Bachelor of Science \\
        % \textbf{PhD} & Doctor of Philosophy \\
    \end{tabular}
\end{table}

\clearpage

% Switch to Arabic numbering after preliminary pages
\pagenumbering{arabic}

% -------------------------
% Chapters
% -------------------------
\chapter{Introduction}
\section{Background}
\todo{Fill this section out}
Introduce the research topic. 

The things in this section will include
\begin{itemize}
    \item Talk about the ISIS research facility
    \item Talk about the Operational Cycle for ISIS (graph too)
    \item Talk about the Datasets, operalog
\end{itemize}

\section{Motivation}
Motivate the research project.


The things in this section will include
\begin{itemize}
    \item Introduce the problem: Auto-categorisation and label inference
    \item Identify the data input, expected output, data shape and explain why this motivates the project
    \item Natural Language Processing
    \item Semantic Similarity
    \item Need for clustering
\end{itemize}

\chapter{Literature Review}

The things in this section will include
\begin{itemize}
    \item Looking at general predictive maintenance 
    \item Looking at general predictive maintenance in industrial applications
    \item Similar pairwise sentence similarity literature
    \item Similar literature in text clustering
    \item Similar literature in specifically sentence clustering in industrial applications
\end{itemize}

\chapter{Technical Background}
Describe the various technical factors required before attempting to understand the methodology

The things in this section will include
\begin{itemize}
    \item Discuss sentence embedding, similarity measures: BERT, RoBERTA, MPNET, XLM, NOMIC.
    \item Dimensional reduction techniques and need for them (UMAP, PCA, t-SNE).
    \item Clustering methods: kmedoids, DBSCAN, DBSCAN*/HDBCAN
    \item Clustering evaluation methods: \todo{I don't remember these off the top of my head}
    \item Maybe briefly touch on Optuna?
\end{itemize}


\chapter{Methodology}
Describe the methods and procedures used.

The things in this section will include.
\begin{itemize}
    \item Explaining data format and data visualisation: wordcloud.
    \item Data cleaning steps, including removing key words such as Ion Source.
    \item Text preprocessing steps (cleaning) and computational challenges (tensorflow).
    \item Choosing the best sentence embedding transformer: MPNET, NOMIC.
    \item Data visualisation (before and after sentence embedding): similarity visualisation, explain unique sentences, token length distribution.
    \item Motivate why clustering in higher dimensions performs worse
    \item UMAP, PCA, t-SNE comparison. Motivate using UMAP.
    \item UMAP hyperparameter optimisation.
    \item Performing clustering with kmedoids, dbscan, hdbscan.
    \item Using optuna.
    \item Evaluation of results and choosing the best model (and arguing why hdbscan is the best by looking at the variance of dbscan and inflexibility of kmedoids)
    \item Touch on the production of a CLI application that allows you to mix and match various parts of the pipeline. Motivate the need for command line tool.
\end{itemize}

\chapter{Results and Discussion}
Describe the results and analyse the results

\begin{itemize}
    \item Analyse the word cloud 
    \item Analyse the sentence embedding results.
    \item Analyse UMAP vs. PCA vs. t-SNE qualitatively and later quantitatively (compared to the clustering)
    \item Anaylse the UMAP hyperparameter optimisation qualitatively, mention that we use Optuna
\end{itemize}

\chapter{Conclusion}
Summarize your findings and suggest areas for future work.

% -------------------------
% References
% -------------------------
\bibliographystyle{plainnat}
\bibliography{refs}  % Uses the external file refs.bib

\end{document}
